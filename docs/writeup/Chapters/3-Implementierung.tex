\chapter{Implementierung Webserver}

\section{Jakarta EE}

Als Referenzimplementierung für dieses Projekt wurde \texttt{Jakarta EE} verwendet.
Jakarta EE ist eine Sammlung von Spezifikationen für Enterprise-Java-Anwendungen.
Es ist eine Weiterentwicklung von Java EE, welches von Oracle entwickelt wurde.
Jakarta EE wird von der Eclipse Foundation entwickelt und ist Open Source.

\todo{Write}

\subsection{Probleme}

\subsubsection*{Abhängigkeiten vom Webserver}

Jakarta EE benötigt einen seperaten Webserver, um die Anwendung auszuführen. 
Die Konfiguration und das Setup sind stark abhängig von dem verwendeten Webserver. 
In diesem Fall wurde \texttt{Wildfly}\footnote{\url{https://www.wildfly.org/} | Abgerufen: 2024-05-10, 12:44 Uhr} verwendet.

Die meisten dieser Webserver sind nicht dazu ausgelegt, eine einzelne Anwendung zu hosten.
Dies macht es schwieriger, die Anwendung in einer Microservice-Architektur zu betreiben.

Um dieses Projekt mit diesem Webserver zu betreiben, ist es notwendig, ein Konfigurations-Skript zu schreiben, welches in den Bauprozess des Docker-Abbilds integriert wird.

\begin{itemize}
    \item Konfiguration und Setup ist stark abhängig von dem verwendeten Webserver (in diesem Fall \texttt{Wildfly})
\end{itemize}

\section{Spring Boot}

\todo{Write}
\section{ASP.NET Core}
\todo{Write}
