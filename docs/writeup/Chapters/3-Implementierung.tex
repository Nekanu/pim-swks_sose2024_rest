\chapter{Implementierung Webserver}

\section{Jakarta EE}
\label{sec:implementation-jakarta}

\textit{Das Jakarta Projekt befindet sich im Ordner:} \textbf{rest-jakarta}\\

Als Referenzimplementierung für dieses Projekt wurde \texttt{Jakarta EE} verwendet.

Diese Implementierung dient als Basis gegenüber der ASP.NET Implementierung (siehe \ref{sec:implementation-aspnetcore})

\subsection{Resourcen}

\textit{Die Resourcen der API sind im} \textbf{resources}\textit{-Package implementiert.} \\

Jede Resource ist eine eigene Klasse und wird mit der \texttt{@Path}-Annotation versehen, welche den Basis-Pfad der Endpunkte in dieser Resource vorgibt.

Alle Endpunkte dieser Resource werden als Methooden in der Klasse implementiert.
Es gibt dafür verschiedene Annotationen, wie:

\begin{itemize}
    \item \texttt{@GET / @POST / ...}: Legt die HTTP-Methode der Anfrage fest.
    \item \texttt{@Path}: Gibt den Pfad nach dem für die Resource deklarierten Basis-Pfad an.
    \item \texttt{@Consumes}: Den \texttt{MediaType} den der Body der Anfrage haben muss.
    \item \texttt{@Produces}: Den \texttt{MediaType} den die Antwort haben wird.
\end{itemize}

Die Methoden-Argumente werden aus den für die Argumente definierten Orten der Anfrage übernommen. Sollte kein Ort angegeben sein, nimmt das Framework dies aus dem Körper der Anfrage.

Alle nötigen Konvertierungen zwischen den Java Objekten und der Serializierung für die Anfragen/Antworten sind implizit und werden von dem Framework übernommen.

\begin{lstlisting}[style=Java, caption={PUT /posts/\{id\} - Deklaration}]
@PUT
@Path("{postId}")
@Consumes(MediaType.APPLICATION_JSON)
@Produces(MediaType.APPLICATION_JSON)
@Secured
public PostDto updatePost(@PathParam("postId") long postId,
                            @Valid final PostDto post)
{...}
\end{lstlisting}

\subsection{Datenbankkommunikation}

\textit{Datenbank-Entities befinden sich im Package} \textbf{model/entities}; \\
\textit{Klassen zur Kommunikation mit der Datenbank befinden sich im} \textbf{repositories}\textit{-Package} \\

Zur Kommunikation mit der Datenbank wird das Hibernate-Framework verwendet.
Die Entities haben Annotationen für die Relationen untereinander und für Datenbank-spezifische Metadaten.

Jede Datenbanktabelle hat eine Implementierung in \texttt{repositories}, welche die nötigen Methoden zur Kommunikation implementiert.

\subsection{Authentifizierung}

\textit{Die Klassen zur Authentifizierung sind im Package} \textbf{middlewares}\\

Eine API-Key Authentifizierung ist nicht vom Framework implementiert, daher muss dies von uns implementiert werden.
Dazu implementieren wir das \texttt{ContainerRequestFilter}-Interface um Anfragen ohne gültigen API-Key als Unauthentifiziert abzuweisen.
Wir deklarieren, dass durch diese Authentifizierung geschützte Endpunkte mit der \texttt{@Secured} Annotation versehen werden.



\section{ASP.NET Core}
\label{sec:implementation-aspnetcore}
\textit{Das ASP.NET Projekt befindet sich im Ordner:} \textbf{rest-aspnet/Rest.AspNet}\\

Als Vergleichsimplementierung wird das ASP.NET Core Framework verwendet. 
\\
Über den Verlauf dieser Sektion werden Vergleiche zur Referenzimplementierung (siehe \ref{sec:implementation-jakarta}) gezogen.

\subsection{Resourcen}

\textit{Die Resourcen der API werden im} \textbf{Controllers}\textit{-Namespace implementiert.} \\

Jede Resource ist auch hier eine eigene Klasse. Diese erweitert hierbei die \texttt{ControllerBase}-Klasse und wird mit dem \texttt{Route}-Attribut versehen, welches analog zur \texttt{@Path}-Anotation in Java den Basispfad der Resource angibt.\\

Auch in ASP.NET sind die einzelnen Endpunkte der Resourcen als Methoden implementiert, und wie auch in Java mit Attributen versehen.

\subsection{Datenbankkommunikation}

\textit{Datenbank-Entities befinden sich im Namespace} \textbf{Model}; \\
\textit{Klassen zur Kommunikation mit der Datenbank befinden sich im} \textbf{Data}\textit{-Namespace} \\

In ASP.Net übernimmt die Datenbankkommunikation \texttt{EntityFramwork} (EF).

Wie auch in Hibernate werden die Models mit Attributen versehen, welche die Datenbankstruktur wiederspiegeln sollen, allerdings wird das Meiste automatisch erkannt (so zum Beispiel die Referenzen zwischen den Entities) und wir benötigen nur ein Attribut für die automatisierte ID-Generation: \texttt{[DatabaseGenerated(DatabaseGeneratedOption.Identity)]}\\

Im Gegensatz zu Hibernate müssen keine Klassen für jede Interaktion mit den Entities geschrieben werden, sondern EF erkennt die Zusammenhänge aus den Entities und stellt \texttt{LINQ}-Methoden\footnote{\url{https://learn.microsoft.com/en-us/dotnet/csharp/linq/}} zu Verfügung.

\subsection{Authentifizierung}

\textit{Die Authentifizierung ist in der Startklasse} \textbf{Program.cs} \textit{implementiert.}\\

Authentifizierung ist deutlich komplexer in ASP.NET als in Jakarta. Dies liegt größtenteils daran,
dass ASP.NET bereits ein komlexes System für Authentifizierung und Authorisierung enthält und einbindet. 
Dementsprechend schwierig ist es, auf diesem System eine API-Key-Authentifizierung zu schreiben. 
Daher haben wir uns dazu entschieden eine externe Implementierung für die API-Key Authentifizierung zu benutzen.

\section{Probleme}

\paragraph{Dependency Management - Maven}

\texttt{Maven} als Build-Managment-Tool ist leider nicht in Java integriert und muss separat installiert werden.
Des Weiteren ist die Konfiguration von Maven deutlich komplexer als die von \texttt{.NET Core}.
In Dotnet Core ist das Dependency-Management in der \texttt{.csproj}-Datei integriert und wird größtenteils (durch die IDE) automatisch verwaltet.
Zusätzlich ist die Konfiguration in der \texttt{.csproj}-Datei deutlich einfacher und übersichtlicher als die in der \texttt{pom.xml}-Datei.

\paragraph{Jakarta - Abhängigkeiten von Webservern}

Im Gegensatz zu ASP.NET oder Spring Boot benötigt Jakarta EE einen seperaten Webserver, um die Anwendung auszuführen. 
Die Konfiguration und das Setup sind stark abhängig von dem verwendeten Webserver. 
In diesem Fall wurde \texttt{Wildfly}\footnote{\url{https://www.wildfly.org/}} verwendet.

Die meisten dieser Webserver sind nicht dazu ausgelegt, eine einzelne Anwendung zu hosten.
Dies macht es schwieriger, die Anwendung isoliert zu betreiben, 
wie etwa in einer Micro-Servicearchitektur mit Docker (wie wir es umsetzen wollten).
Wir haben dazu das offizielle \texttt{Wildfly} Docker Image benutzt,
 allerdings hat auch dieses seine Schwächen.
Alle Konfiguration bezüglich Datenbanken lässt sich nur in Wildfly vornehmen.
Selbst wenn die Applikation die Datenbanken selbst für sich konfiguriert, wird diese Konfiguration von Wildfly überschrieben.
Dies bedeutet, dass diese Service-Referenzen nur in Wildfly vorgenommen werden können.
Wir haben im Rahmen dieses Projektes dies nicht automatisiert (für das Docker-Deployment) umsetzen können
und daher sind wir von der ursprünglich geplanten MariaDB-Datenbank auf die In-Memory-Datenbank H2 umgestiegen.
