\chapter{Implementierung Webserver}

\section{Jakarta EE}
\label{sec:implementation-jakarta}

Als Referenzimplementierung für dieses Projekt wurde \texttt{Jakarta EE} verwendet.

Diese Implementierung dient als Basis gegenüber der ASP.NET Implementierung (siehe \ref{sec:implementation-aspnetcore})

\subsection{Projektstruktur}

Das Jakarta-Projekt gliedert sich in folgende Packages:
\begin{itemize}
    \item \textbf{middlewares}: Beinhaltet die Anfrage-Präprozessoren (hier die Authentifizierung)
    \item \textbf{model}: Beinhaltet die \ac{DTO}s und die Database-Entities.
    \item \textbf{repositories}: Beinhaltet die Klassen zur Interaktion mit der Datenbank
    \item \textbf{resources}: Beinhaltet die Implementation der Resourcen der API
\end{itemize}

\subsection{Resourcen}

Die Resourcen der API werden in dem \texttt{resources}-Package implementiert.

Jede Resource ist eine eigene Klasse und wird mit der \texttt{@Path}-Annotation versehen, welche den Basis-Pfad der Endpunkte in dieser Resource vorgibt.

Alle Endpunkte dieser Resource werden als Methooden in der Klasse implementiert.
Es gibt dafür verschiedene Annotationen, wie:

\begin{itemize}
    \item \texttt{@GET / @POST / ...}: Legt die HTTP-Methode der Anfrage fest.
    \item \texttt{@Path}: Gibt den Pfad nach dem für die Resource deklarierten Basis-Pfad an.
    \item \texttt{@Consumes}: Den \texttt{MediaType} den der Body der Anfrage haben muss.
    \item \texttt{@Produces}: Den \texttt{MediaType} den die Antwort haben wird.
\end{itemize}

Die Methoden-Argumente werden aus den für die Argumente definierten Orten der Anfrage übernommen. Sollte kein Ort angegeben sein, nimmt das Framework dies aus dem Körper der Anfrage.

Alle nötigen Konvertierungen zwischen den Java Objekten und der Serializierung für die Anfragen/Antworten sind implizit und werden von dem Framework übernommen.

\begin{lstlisting}[style=Java, caption={PUT /posts/\{id\} - Deklaration}]
@PUT
@Path("{postId}")
@Consumes(MediaType.APPLICATION_JSON)
@Produces(MediaType.APPLICATION_JSON)
@Secured
public PostDto updatePost(@PathParam("postId") long postId,
                            @Valid final PostDto post)
{...}
\end{lstlisting}

\subsection{Datenbankkommunikation}

Zur Kommunikation mit der Datenbank wird das Hibernate-Framework verwendet.

Die Entities sind in \texttt{model/entities} implementiert und haben Annotationen für die Relationen untereinander.

Jede Datenbanktabelle hat eine Implementierung in \texttt{repositories}, welche die nötigen Methoden zur Kommunikation implementiert.

\subsection{Authentifizierung}

Eine API-Key Authentifizierung ist nicht vom Framework implementiert, daher muss dies von uns implementiert werden.
Dazu implementieren wir das \texttt{ContainerRequestFilter}-Interface um Anfragen ohne gültigen API-Key als Unauthentifiziert abzuweisen.
Wir deklarieren, dass durch diese Authentifizierung geschützte Endpunkte mit der \texttt{@Secured} Annotation versehen werden.



\section{ASP.NET Core}
\label{sec:implementation-aspnetcore}
Als Vergleichsimplementierung wird das ASP.NET Core Framework verwendet. 
\\
Über den Verlauf dieser Sektion werden Vergleiche zur Referenzimplementierung (siehe \ref{sec:implementation-jakarta}) gezogen.

\subsection{Projektstruktur}

\subsection{Resourcen}

\subsection{Datenbankkommunikation}

\subsection{Authentifizierung}

\todo{Write}

\section{Fazit}

\paragraph{Jakarta - Abhängigkeiten von Webservern}

Im Gegensatz zu ASP.NET oder Spring Boot benötigt Jakarta EE einen seperaten Webserver, um die Anwendung auszuführen. 
Die Konfiguration und das Setup sind stark abhängig von dem verwendeten Webserver. 
In diesem Fall wurde \texttt{Wildfly}\footnote{\url{https://www.wildfly.org/}} verwendet.

Die meisten dieser Webserver sind nicht dazu ausgelegt, eine einzelne Anwendung zu hosten.
Dies macht es schwieriger, die Anwendung in einer Microservice-Architektur zu betreiben.

Um dieses Projekt mit diesem Webserver zu betreiben, ist es notwendig, ein Konfigurations-Skript zu schreiben, welches in den Bauprozess des Docker-Abbilds integriert wird.

\begin{itemize}
    \item Konfiguration und Setup ist stark abhängig von dem verwendeten Webserver (in diesem Fall \texttt{Wildfly})
\end{itemize}

\todo{Write}
