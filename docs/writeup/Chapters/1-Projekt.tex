\chapter{Projekt}

\section{Zielsetzung}

Gegenstand des Projekts war die Implementierung und das Testen einer REST-API sowie die Bewertung der eingesetzten Werkzeuge.

Die primäre Implementierung der API wurde mit Jakarta umgesetzt.
Um eine fundierte Bewertung des Frameworks zu ermöglichen, 
wurde eine alternative Implementierung mit ASP.NET umgesetzt.

Neben der Implementierung der API bestand ein wesentlicher Bestandteil des Projekts 
in der Automatisierung von Tests zur Sicherstellung der Korrektheit und Zuverlässigkeit der API.
Dies umfasste sowohl funktionale Tests, als auch Load Tests. 
Besonders sollte hier auf die Technologien zur Automatisierung solcher Tests konzentriert werden.

    
\todo{Write}

\section{Projektbeiträge}

\begin{itemize}
    \item \textbf{Spezifikation}: Alle
    \item \textbf{Implementation}: Joshua Nestler
    \item \textbf{Testing - Postman}: Arne Kreuz
    \item \textbf{Testing - JMeter}: Moritz Schönenberger
    \item \textbf{CI/CD}: Joshua Nestler
\end{itemize}

\section{Verwendete Technologien}

\begin{table}[!h]
    \centering
    \begin{tabular}{|l|l|l|}
        \hline
        \textbf{Technologie} & \textbf{Version} & \textbf{Link} \\
        \hline
        
        Java & 21 & \url{https://www.java.com/} \\
        Maven & 3.8.5 & \url{https://maven.apache.org/} \\
        Jakarta EE & 11.0.0  & \url{https://jakarta.ee/} \\
        Wildfly & 32.0.0 & \url{https://www.wildfly.org/} \\
        \hline
        
        .NET & 8.0 & \url{https://dotnet.microsoft.com/} \\
        ASP.NET Core & 8.0 & \url{https://dotnet.microsoft.com/apps/aspnet} \\
        \hline
        
        H2 & 2.2.224 & \url{https://www.h2database.com/} \\
        \hline
        
        JMeter & 5.4.1 & \url{https://jmeter.apache.org/} \\
        Postman & 10.23.4 & \url{https://www.postman.com/} \\
        \hline
        
        Docker & 27 & \url{https://www.docker.com/} \\
        IntelliJ IDEA & 2024 & \url{https://www.jetbrains.com/idea/} \\
        Rider & 2024 & \url{https://www.jetbrains.com/rider/} \\
        \hline

        Git & 2.45.0 & \url{https://git-scm.com/} \\
        GitHub & - & \url{https://github.com/} \\
        GitHub Actions & - & \url{https://github.com/features/actions} \\
        \hline

        OpenAPI & 3.0.3 & \url{https://www.openapis.org/} \\
        LaTeX & 2023 & \url{https://www.latex-project.org/} \\
        \hline
    \end{tabular}

    \caption{Verwendete Technologien}
    \label{tab:technologies}
\end{table}
