\chapter{Fazit}

Im Rahmen des Projekts wurde erfolgreich eine REST-API implementiert und getestet.\\


Zunächst wurde die API unter Einsatz von OpenAPI konzeptioniert und definiert.
Die hierbei entstandene Spezifikation führt alle Endpunkte sowie ihre Antworten auf.
Basierend auf der OpenAPI-Spezifikation wurde die API anschließend in Jakarta und ASP.NET implementiert.
Zur Implementierung unter Jakarta wurde Wildfly als Server eingesetzt.
Die Korrektheit der Implementierung wurde unter Einsatz von Postman entsprechend der Spezifikation geprüft.
Zusätzlich wurde ein JMeter-Testplan zur Erfassung von Leistungsdaten entwickelt.


\todo[inline]{kurzbewertung implementierung}

Postman erweist sich als ein benutzerfreundliches und leistungsstarkes Werkzeug für API-Tests,
allerdings kann das Always-Online-Modell störend sein, da es eine ständige Abhängigkeit von der Postman API mit sich bringt.
Zudem schränkt das Bezahlmodell die Anzahl der täglichen Testausführungen in der kostenlosen Version ein,
was bei umfangreicheren Projekten zu erheblichen Einschränkungen führen kann. 
Angesichts dessen wäre es sinnvoll, 
für zukünftige Entwicklungen auch alternative Tools in Betracht zu ziehen. 

JMeter kann grundsätzlich trotz einiger Fallstricke und Einschränkungen empfohlen werden.
Das Werkzeug erfordert eine explizite Einarbeitung,
kann nach dieser jedoch für viele Testszenarien eingesetzt werden.
Die Generierung des HTML-Reports funktioniert out of the box
und bereitet die erfassten Leistungsparameter sinnvoll auf.
Daten der verschiedenen Sampler können direkt im HTML-Report ein- und ausgeblendet werden,
sodass der Report einen guten Überblick bietet.


\todo{Write}
\section{Zusammenfassung}

\todo{Write}
\section{Lessons Learned}

\todo{Write}
\section{Ausblick}

\todo{Write}
