%************************
%*		Tabellen 		*
%************************

\section{Tabellen}
\label{sec:Tabellen}

\subsection{Einfache Tabelle}
In LaTeX lassen sich Tabellen unterschiedlicher Ausprägung einfach erzeugen. Das allgemeine Format einer Tabelle sieht aus wie folgt:

\begin{lstlisting}[caption={Allgemeines Format}]
\begin{table}
	\caption{BESCHRIFTUNG}
	\begin{tabular}{FORMATIERUNG}
		TABELLENINHALT
	\end{tabular}
\end{table}
\end{lstlisting}

Eine Beispieltabelle (Tabelle \ref{tab:beispieltabelle1}) könnte also so aussehen:

\begin{lstlisting}[caption={Tabelle \ref{tab:beispieltabelle1}}]
\begin{table}
	\caption{Beispiel 1}
	\begin{tabular}{lrcr}
		\toprule
		\textbf{Name} & \textbf{Vorname} & \textbf{Matrikelnummer} & \textbf{Lieblingsspeise}\\
		\midrule
		Jackson & Michael & 123456 & Erdbeereis \\
		Springsteen & Bruce & 234567 & Schwedisches Lakritz \\
		Bach & Anna, Magdalena & 3456789 & Frankfurter Kranz \\
		Schumann & Clara & 4567890 & Bisquitt\"ortchen \\
		\bottomrule
	\end{tabular}
	\label{tab:beispieltabelle1}
\end{table}
\end{lstlisting}

Mit \lstinline|\caption{Beispiel 1}| bekommt unsere Tabelle eine Beschriftung am Tabellenkopf. \lstinline{l|r|c|r} legt die Textausrichtung der einzelnen Spalten fest: \lstinline|l| bedeutet linksausgerichtet, \lstinline|r| rechtsausgerichtet und \lstinline|c| zentriert. Durch \lstinline{|} werden Spaltenlinien gezogen. \lstinline|\toprule|, \lstinline|\midrule| und \lstinline|\bottomrule| erzeugen Kopf-, Mittel- und Abschlusslinie in der Tabelle. Als Spaltentrenner wird das \lstinline{&} genutzt, Zeilentrenner ist der doppelte Backslash (\lstinline|\\|). Am Ende kann die Tabelle auch mit einem Label versehen werden (\lstinline|\label{tab:beispieltabelle1}|), über welches diese referenziert wird.

%\begin{center}
\begin{table}[b]
	
	\caption{Beispiel 1}
	\begin{tabular}{lrcr}
		\toprule
		\textbf{Name} & \textbf{Vorname} & \textbf{Matrikelnummer} & \textbf{Lieblingsspeise}\\
		\midrule
		Jackson & Michael & 123456 & Erdbeereis \\
		Springsteen & Bruce & 234567 & Schwedisches Lakritz \\
		Bach & Anna, Magdalena & 3456789 & Frankfurter Kranz \\
		Schumann & Clara & 4567890 & Bisquittörtchen \\
		\bottomrule
	\end{tabular}
	\label{tab:beispieltabelle1}
\end{table}
%\end{center}

\subsection{Erweiterte Tabellenbefehle}
Um Tabellen in LaTeX flexibler zu gestalten gibt es weitere Befehle bzw. zusätzliche Pakete, die einem das Leben leichter machen (Tabelle \ref{tab:beispieltabelle2}). Hierzu ein weiteres Beispiel:

\begin{lstlisting}[caption={Tabelle \ref{tab:beispieltabelle2}}]
\begin{table}
	\centering
	\caption{Beispiel 2}
	\begin{tabular}{lll}
		\hline
		Author & Title & Year \\
		\hline
		\hline
		\multirow{3}{*}{Stanislav Lem} & Solaris & 1961 \\
 			& Roboterm\"archen & 1967 \\
 			& Der futurologische Kongress & 1971 \\
		\hline
		\multirow{3}{*}{Isaac Asimov} & Ich, der Robot & 1952 \\
 			& Der Tausendjahresplan & 1966 \\
 			& Doctor Schapirows Gehirn & 1988 \\
		\hline
	\end{tabular}
\label{tab:beispieltabelle2}
\end{table}
\end{lstlisting}

Mit \lstinline|\centering| wird die Tabelle zentriert ausgerichtet, analoge Befehle für rechts- bzw. linksausrichtung sind z.B. \lstinline|\raggedleft| und \lstinline|\raggedright|. \\

Eine weitere Form der Tabellen ist das Package \textit{tabularx}, das variable Spaltenbreiten unterstützt, und \textit{booktabs}, welches mit horizontalen Linien besser arbeiten kann.
\begin{table}
	\centering
	\caption{So sollte man es nicht machen! Beispiel für einen schlechten Tabellenstil}
	\begin{tabular}{|l|l|l|}
		\hline
		Author & Title & Year \\
		\hline
		\hline
		\multirow{3}{*}{Stanislav Lem} & Solaris & 1961 \\
 			& Robotermärchen & 1967 \\
 			& Der futurologische Kongress & 1971 \\
		\hline
		\multirow{3}{*}{Isaac Asimov} & Ich, der Robot & 1952 \\
 			& Der Tausendjahresplan & 1966 \\
 			& Doctor Schapirows Gehirn & 1988 \\
		\hline
	\end{tabular}
\label{tab:beispieltabelle2}
\end{table}